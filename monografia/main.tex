\documentclass[a4paper,11pt]{article}
\usepackage[left=3cm,top=3cm,right=2cm,bottom=2cm]{geometry}
\usepackage[brazilian, english]{babel}
\usepackage[utf8]{inputenc}
\usepackage[numbers]{natbib}
\usepackage{epigraph}
\usepackage{indentfirst}
\usepackage{graphicx}
\usepackage{wrapfig}
\usepackage{setspace}
\usepackage[hidelinks]{hyperref}
%\usepackage{xcolor}
\usepackage[dvipsnames]{xcolor}
\usepackage{adjustbox}
\newcommand{\tabitem}{~~\llap{\textbullet}~~}
\newcommand*{\SignatureAndDate}[4]{%
	\parbox{7cm}{
      \centering
      \rule{6cm}{1pt}\\
       #1
       
       #2
    }
    \hfill
\parbox{7cm}{
      \centering
      \rule{6cm}{1pt}\\
       #3
       
       #4
    }
}%
\usepackage{float}

\floatstyle{ruled}
\newfloat{program}{thp}{lop}
\floatname{program}{Program}

\onehalfspacing

\title{Relatório de Estágio III}

\renewcommand{\familydefault}{\sfdefault}

\begin{document}
\selectlanguage{brazilian}

%%% CAPA %%%

\begin{titlepage}

\begin{wrapfigure}[2]{l}{0.2\textwidth}
	\label{Logo UFABC}
	\vspace{-1\baselineskip}
	\centering
	\includegraphics[width=0.25\textwidth]{images/Logo_UFABC}
\end{wrapfigure}

\uppercase{Universidade Federal do ABC}

\uppercase{Bacharelado em Ciência da Computação}

\vfill
\begin{center}

\uppercase{\textbf{Projeto de Graduação em Computação}}

\vfill

\uppercase{Bruno Cesar Porto de Arruda}
\vspace{1cm}

Orientador: Prof. Dr. Vladimir Moreira Rocha

\vfill

Santo André -- SP

2019
\end{center}
\end{titlepage}

%%% FIM DA CAPA %%%

%%% FOLHA DE ROSTO %%%

\begin{titlepage}
\begin{center}
\uppercase{\textbf{Bruno Cesar Porto de Arruda}}

\vfill

\uppercase{\textbf{Um sistema distribuído com permissão de acesso a prontuários de pacientes por meio de Smart Contracts}}
\end{center}

\vfill

\hfill \begin{minipage}{0.5\textwidth}
Trabalho submetido à Universidade Federal do ABC como parte dos requisitos para a conclusão do Bacharelado em Ciência da Computação.
\vspace{1cm}

Orientador: Prof. Dr. Vladimir Moreira Rocha
\end{minipage}

\vfill

\begin{center}
Santo André -- SP

2019
\end{center}
\end{titlepage}

%%% FIM DA FOLHA DE ROSTO %%%

\begin{center}
\uppercase{\textbf{Dedicatória}}
\end{center}
	xxx.


\newpage
\begin{center}
\uppercase{\textbf{Agradecimentos}}
\end{center}

\noindent	Ofereço meus sinceros agradecimentos:

\vspace{1cm}


%%% ABSTRACT - PORTUGUÊS %%%
\newpage
\begin{abstract}

\noindent FAZER NO FINAL.\\

\noindent \textbf{Palavras-Chave:} Blockchain; Ethereum; Contratos Inteligents; CBA.
\end{abstract}

%%% ABSTRACT - INGLÊS %%%

\newpage
\selectlanguage{english} 
\begin{abstract}
\noindent FAZER NO FINAL. \\

\noindent \textbf{Keywords:} Blockchain; Ethereum; Smart Contracts; ABE.
\end{abstract}

\selectlanguage{brazilian} 

%%% SUMÁRIO %%%
\newpage
\tableofcontents

%%% LISTA DE FIGURAS %%%
\newpage
\listoffigures

%%% LISTA DE TABELAS %%%

% \newpage
% \listoftables

% -------------------------------------------------------------------- %
\newpage
\section{Introdução}

A tecnologia Blockchain~\cite{nakamoto2008bitcoin}...	

\subsection{Objetivo geral}

Criar um sistema, com base em contratos inteligentes executados em uma Blockchain, que permita o acesso a prontuários eletrônicos dos pacientes via políticas de acesso baseadas em atributos.

\subsection{Objetivos específicos}

\begin{itemize}

\item Criar uma taxonomia de permissões no contexto de saúde.

\item Analisar como funciona a criptografia baseada em atributos (\textit{attribute-based encryption}, em inglês). 

\item Analisar como funcionam a tecnologia Blockchain {\color{red}escolher a tecnologia} e os contratos inteligentes. 

\item Implementar os contratos inteligentes para dar acesso aos prontuários eletrônicos utilizando a taxonomia e a criptografia baseada em atributos.

\item Implantar e executar os contratos inteligentes em uma arquitetura Blockchain.

\end{itemize}

\subsection{Justificativa}

% -------------------------------------------------------------------- %
\newpage
\section{Fundamentação Teórica}

\subsection{Criptografia baseada em atributos (CBA)}

\subsection{Blockchain Ethereum}

\subsection{Contratos inteligentes no Ethereum}

% -------------------------------------------------------------------- %
\newpage
\section{Sistema Proposto}

\begin{itemize}
    \item {\color{red} Assuma nesta seção que os conceitos de blockchain, Ethereum, contratos inteligentes e criptografia baseada em atributos já foram definidos e explicados.}
    
    \item {\color{red}Cada parágrafo deve ter em torno de 10 linhas}
    
    \item {\color{red}Não mostrar código.}
    
\end{itemize}

\subsection{Visão Geral}

{\color{ForestGreen}Explicar que o sistema é composto por X componentes (1 parágrafo).} \\

{\color{ForestGreen}Explicar a função de cada componente (1-2 parágrafos para cada um).} \\

{\color{ForestGreen}Figura dos principais componentes em alto nível: programa cliente (exemplo celular ou notebook), servidor de armazenamento de arquivos; blockchain ethereum; contratos inteligentes; servidor de chaves de permissões.} \\

{\color{ForestGreen}Explicar um cenário de uso passo-a-passo (exemplo, milestone 1, mas com permissões direcionadas ao contexto de saúde, por exemplo um paciente quer dar permissão de acesso a médicos cardiologistas) (2 parágrafos)}.

\subsection{Taxonomia de permissões}

{\color{ForestGreen}Explicar que o servidor de permissões deverá ser responsabilidade de uma organização apta para entregar as permissões online (ex. Ministério de Saúde ou algúm conselho federal/regional de medicina) (1 parágrafo)}. \\

{\color{ForestGreen}Explicar que a autenticação de uma pessoa para obter a permissão deverá ser realizada de forma externa ao sistema proposto (1 parágrafo).} \\

{\color{ForestGreen}Explicar que foi realizada uma taxonomia das possíveis permissões que seriam utilizadas no sistema para o contexto médico. (3 parágrafos explicando como foi realizado o levantamento, me lembro que tinha um artigo e uma lista gigante que foi filtrada por quantidade de médicos).} \\

{\color{ForestGreen}Figura da taxonomia}. \\

{\color{ForestGreen}Explicar que no sistema proposto existe um servidor de atributos ao qual se pedem as chaves públicas/privadas para realizar a encriptação/decriptação (1 parágrafo).} \\

{\color{ForestGreen}Explicar como é realizado o passo-a-passo para o pedido/entrega das chaves (note que aqui a descrição é muito mais profunda que o que foi mencionado na visão geral) (1 parágrafo).}

\subsection{Contratos inteligentes com CBA}

\subsection{Arquitetura Blockchain}

% -------------------------------------------------------------------- %
\newpage
\section{Trabalhos Relacionados}

\subsection{Sistemas de saúde com Blockchain e CBA}

\subsection{Sistemas de saúde com CBA}
\label{sub:sec:saude-cba}

\subsection{Sistemas gerais com CBA}

Caso não hajam trabalhos na seção~\ref{sub:sec:saude-cba}.

% -------------------------------------------------------------------- %
\newpage
\section{Resultados Experimentais}



% -------------------------------------------------------------------- %
\newpage
\section{Conclusões e Trabalhos futuros}

\bibliography{doc}
\bibliographystyle{plainnat}


\end{document}